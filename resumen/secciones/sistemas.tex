\subsection{Sistemas de Ecuaciones Lineales}
Un \textbf{sistema de ecuaciones lineales} o sistema lineal es una colección de dos o mas ecuaciones lineales (nótese, uso no análogo a afín/lineal) que involucra el mismo conjunto de variables. Una \textbf{solución} a tal sistema es una asignación de valores a las variables de forma de que todas se encuentren satisfechas.\\

\textit{Nota:} Formalmente, se define como forma general:\\
$a_{11} x_1 + a_{12} x_2  + \cdots + a_{1n} x_n  = b_1 \\
    a_{21} x_1 + a_{22} x_2  + \cdots + a_{2n} x_n  = b_2 \\
    \ \ \vdots\\
    a_{m1} x_1 + a_{m2} x_2  + \cdots + a_{mn} x_n  = b_ m$\\
    Donde $x_1, x_2,\ldots,x_n$ son las incógnitas, $a_{11},a_{12},\ldots,a_{mn}$ los coeficientes, y $b_1,b_2,\ldots,b_m$ términos constantes.\\

    Aparte de esta forma, un sistema lineal puede ser representado como una combinación lineal de una ecuación vectorial, una ecuación de matrices (por equivalencia) y una representación geométrica de \textit{n-}planos.
\subsubsection{Interpretación Geométrica}
Un sistema de ecuaciones lineales con dos incógnitas y dos ecuaciones pueden representarse como \textbf{dos rectas en el plano}, con sus respectivas intersecciones correspondiendo a las soluciones al sistema.
\textit{Nota:} Para tres variables, cada ecuación linear determina un plano en espacio tridimensional, y para \textit{n} variables, cada ecuación lineal determina un hiperplano (un subespacio con una dimensión una menor que la de su espacio) en un espacio \textit{n-}dimensional.
\subsubsection{Soluciones}
Según la cantidad de soluciones, un sistema de ecuaciones lineales (en este caso. $2\times2$) puede ser clasificado en:\\
\textit{Compatible determinado:} Una solución, es decir, las rectas son secantes entre ellas (un punto de intersección).\\
\textit{Compatible indeterminado:} Infinitas soluciones, es decir, las rectas son coincidentes.\\
\textit{Incompatible:} Ninguna solución, ocurre cuando las rectas son paralelas (igual pendiente).\\