\section{Números}
\def\svgwidth{\columnwidth}
\input{imagenes/conjuntos-numericos.pdf_tex}\\
\subsection{Naturales ($\mathbb{N}$)}
\textit{MCM (mínimo común múltiplo):}
menor Nº  entero positivo que es divisible por cada uno de los números (sin resto).\\

\textit{MCD (máximo común divisor):}
mayor Nº entero positivo que divide cada uno de los números (sin resto).\\

\subsection{Enteros ($\mathbb{Z}$)}
$a < b \rightarrow -a > b$\\
$a < 0 \rightarrow -a > 0$\\
$|x - y| = |y - x|$\\
$ a < b \land c < b \rightarrow b + d > a + c $\\

\subsection{Racionales ($\mathbb{Q}$)}
\textit{Proporcionalidad Directa: (lineal)}\\
$ y = kx \leftrightarrow y \propto x $ con $k = \frac{y}{x}$\\

\textit{Proporcionalidad Inversa: (híperbola rectangular)}\\
$ y = \frac{k}{x} \leftrightarrow k = xy$\\

\subsection{Decimales}
$ 0,\overline{1} = \frac{1}{9} $\\
$ 0,\overline{36} = \frac{36}{96} $\\
$ 1,23\overline{4} = \frac{1234 -123}{900} $\\

\subsection{Imaginarios y Complejos ($\mathbb{C}$)}
$ z = a + bi $ con $ i = \sqrt{1} $\\
$\mid z \mid = \sqrt{a^2 + b^2}$\\

\subsection{Potencias}
$ (-a)^2 = a^2; -a^2 = -(a^2)$\\
$b^{m + n} = b^m \cdot b^n \\
\left(b^m\right)^n = b^{m \cdot n} \\
(b \cdot c)^n = b^n \cdot c^n$\\
$ 0^0 \notin \mathbb{R}$\\

No son conmutativas (e.j $2^3 = 8 \neq 3^2 = 9$) ni asociativas (e.j $(2^3)^4) = 8^4 \neq 2^{(3^4)} = 2^{81}$).\\

Sin paréntesis el orden de operación es de arriba hacia abajo (o dextro-asociativo):\\
$b^{p^q} = b^{\left(p^q\right)} \not\equiv \left(b^p\right)^q = b^{p q}$\\

\subsection{Raíces}
$ a ^ {\frac{p}{q}} \equiv \sqrt[q]{a^p} $\\
$ \sqrt[p]{a^q} \equiv (\sqrt[p]{a})^q $\\
$\sqrt{x^2} = \mid x \mid$\\
$\sqrt[p]{\sqrt[q]{a}} = \sqrt[pq]{a}$\\
Para radicandos $a, b$ positivos:
$\sqrt[n]{ab} \equiv \sqrt[n]{a} \sqrt[n]{b}$\\
$\sqrt[n]{\frac{a}{b}} \equiv \frac{\sqrt[n]{a}}{\sqrt[n]{b}}$\\
Sutilezas con radicandos negativos:
$\sqrt{-1}\times\sqrt{-1} \neq \sqrt{-1 \times -1} = 1,\quad$ sino $\quad\sqrt{-1}\times\sqrt{-1} = i \times i = i^2 = -1.$\\

\subsection{Logaritmos ($b > 0$)}
$\log_b(x) = y \quad \leftrightarrow \quad b^y = x.$\\
$\log_b(xy) = \log_b x + \log_b y$\\
$\log_b \!\frac{x}{y} = \log_b x - \log_b y$\\
$\log_b\left(x^p\right) = p \log_b x$\\
$\log_b \sqrt[p]{x} = \frac{\log_b x}{p}$\\
$\log_b x = \frac{\log_k x}{\log_k b}$\\
$\log_b \frac{1}{x} = -\log_b x$

\subsection{Interés Simple y Compuesto}
\textbf{Interés Simple:} $C_f = C_i(1+n\cdot i\%)$\\
\textbf{Interés Compuesto:} $C_f = C_i (1 + i \%)^n$\\

Con $C_i$ capital inicial, $C_f$ capital final, $i\%$ tasa de interés y $n$ el número de periodos.\\

\textit{Advertencia:} Es clave convertir los períodos de tiempo si es necesario (\textit{e.j} $1$ año $= 12$ meses)\\