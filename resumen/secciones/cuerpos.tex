\subsection{Cuerpos Geométricos}
Se denominan \textbf{cuerpos o sólidos de revolución} a aquellos objetos geométricos que pueden obtenerse mediante la rotación de una figura plana alrededor de una recta denominada eje. (en un espacio tridimensional, \textit{e.j} un cilindro recto por rotación de un rectángulo, una esfera por un semicirculo, un cono por un triángulo rectángulo y un cono truncado por un trapecio rectángulo.)\\

Por el otro lado, un \textbf{cuerpo de traslación} es uno obtenido mediante la traslación de una figura plana. (\textit{e.j} el prisma recto, generado por un polígono trasladado perpendicular a su plano; el cilindro recto generado por la traslación de un círculo).
\subsubsection{Prisma}
Un prisma es un cuerpo geométrico con dos caras paralelas y congruentes llamadas bases, con la restantes caras laterales siendo paralelogramos. Se nombran según los polígonos de sus bases. \\
\begin{equation*}
    \begin{split} 
    A_{\text{lateral}} &= h \cdot P_b\\
    A_{\text{total}} &= 2\cdot A_b + A_l\\
    V &= A_b\cdot h
    \end{split}
\end{equation*}

\subsubsection{Pirámide}
Las pirámides son cuerpos que tienen como base un polígono cualquiera y sus caras laterales concurren en un punto llamado cúspide. Se nombran según el polígono de su base.\\
\begin{equation*}
    \begin{aligned} 
    V = \frac{1}{3}A_b\cdot h
    \end{aligned}
\end{equation*}
\subsubsection{Cilindro}
El cilindro recto es el cuerpo generado al girar un rectángulo en torno a la recta que contiene a uno de sus lados, o bien al traslador un círculo de forma perpendicular a un plano que contiene la base. Este cuerpo queda limitado por una superficie curva llamado manto (label) y dos supericies planas circulares, llamadas bases.

\begin{equation*}
    \begin{split} 
    A_{\text{lateral}} &= 2\pi rh\\
    A_{\text{basal}} &= \pi r^2\\
    A_{\text{total}} &= 2\pi rh + 2\pi ^2 \equiv 2\pi r(h+r)\\
    V &= \pi r^2h
    \end{split}
\end{equation*}
\subsubsection{Cono}
Un cono recto es el cuerpo generado al girar un triángulo rectángulo en torno a la recta que contiene uno de sus catetos. La hipotenusa de tal triángulo se llama generatriz. ($g$)

\begin{equation*}
    \begin{split} 
        &g = \sqrt{r^2 + h^2}\\
        A_{\text{total}} &= A_b + A_l = \pi r^2 + \pi rg \\
        &= \pi r(r + \sqrt{r^2 + h^2}
    \end{split}
\end{equation*}
\begin{equation*}
    \begin{aligned} 
        V = \frac{1}{3}\pi r^2h
    \end{aligned}
\end{equation*}
\subsubsection{Esfera}
La esfera es el cuerpo generado al girar un semicírculo en torno a la recta que contiene al díametro. También puede ser descrito como un conjunto de todos los puntos en el espacio cuya distancia a un punto fijo, llamado centro, es menor o igual a su radio.

\begin{equation*}
    \begin{split} 
        A &= 4\pi r^2\\
        V &= \frac{4}{3} \pi r^3
    \end{split}
\end{equation*}