\subsection{Dominio y Recorrido}
El \textit{dominio} de definición de una función es el conjunto de valores de entrada, o argumentos, para la cual una cierta función está definida. Análogamente, el conjunto de valores de salida, es denominado \textit{rango o imagen}, que es un subconjunto del codominio de la función. Gráficamente, el dominio está representado por el eje x de un plano cartesiano, y el rango por el eje y.\\

Para una función $f: X \rightarrow Y$, donde el conjunto $X$ es el dominio, e $Y$ el codominio, el rango está definido por $\left\{ f(x) | x \in X \right\}$, y siempre es un subconjunto de $Y$. Cuando el rango es todo el conjunto del codominio, la función es sobreyectiva.\\

El rango también se puede encontrar en el dominio de la función inversa, es decir, invirtiendo la variable independiente con la dependiente, y resolviendo el dominio.\\
